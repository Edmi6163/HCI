\documentclass[a4paper,12pt]{article}

\usepackage{amsmath}
\usepackage{graphicx}
\usepackage{hyperref}
\usepackage{geometry}
\geometry{a4paper, margin=1in}

\title{Analisi del Test di Krug}
\author{Francesco Mauro, Riccardo Oro}
\date{Luglio 2024}

\begin{document}

\maketitle

\begin{abstract}
Questo documento presenta i risultati del Test di Krug condotto su tre soggetti come parte di un esame universitario per valutare l'usabilità di un sito web.
\end{abstract}

\tableofcontents

\newpage

\section{Introduzione}
Il Test di Krug è un metodo di test di usabilità progettato per identificare problemi nell'interfaccia utente di un sito web. Consiste nell'osservare i soggetti mentre eseguono compiti specifici e raccogliere dati qualitativi e quantitativi per migliorare l'esperienza utente.

\section{Descrizione del Test}
Il Test di Krug prevede i seguenti passaggi:
\begin{enumerate}
    \item Selezione dei soggetti.
    \item Preparazione dei compiti per i soggetti.
    \item Osservazione e registrazione delle interazioni dei soggetti con il sito web.
    \item Analisi dei dati raccolti per identificare problemi di usabilità.
\end{enumerate}

\section{Soggetti}
Sono stati selezionati tre soggetti per il test. I loro dettagli sono i seguenti:
\begin{itemize}
    \item Soggetto 1: [Utente 1]
    \item Soggetto 2: [Utente 2]
    \item Soggetto 3: [Utente 3]
\end{itemize}

\section{Compiti}
I seguenti compiti sono stati assegnati a ciascun soggetto:
\begin{itemize}
    \item Compito 1: [Leggere gli articoli]
    \item Compito 2: [Uso del sistema di messaggistica]
    \item Compito 3: [Visualizzazione dei dati]
\end{itemize}

\section{Risultati}
\subsection{Soggetto 1}
\subsubsection{Leggere gli articoli}
\begin{itemize}
    \item Osservazioni: [Completato]
    \item Tempo impiegato: [$\simeq$ 20 secondi]
    \item Errori: []
    \item Commenti: []
\end{itemize}
\subsubsection{Uso del sistema di messaggistica}
\begin{itemize}
    \item Osservazioni: [Completato]
    \item Tempo impiegato: [circa 30 secondi] 
    \item Errori: [0]
    \item Commenti: [nome utente inserito nel campo room ]
\end{itemize}
\subsubsection{Visualizzazione dei dati}
\begin{itemize}
    \item Osservazioni: [Fallito]
    \item Tempo impiegato: [1 minuto]
    \item Errori: [1]
    \item Commenti: [L'utente non è riuscito a trovare l'apposito pulsante che l'avrebbe rendirizzato verso l'apposita pagina ,poichè non ha trovato il tasto per abilitare modalità t]
\end{itemize}

\subsection{Soggetto 2}
\subsubsection{Leggere gli articoli}
\begin{itemize}
    \item Osservazioni: [Completato]
    \item Tempo impiegato: [$\simeq$ 12 secondi]
    \item Errori: [0]
    \item Commenti: [L'utente non ha riscontrato problemi nel leggere gli articoli presentati nella home]
\end{itemize}
\subsubsection{Uso de sistema interno di messaggisticaggistica}
\begin{itemize}
    \item Osservazioni: [Fallito]
    \item Tempo impiegato: [circa 40 secondi]
    \item Errori: [1]
    \item Commenti: [L'Utente ha inserito nel campo'Room' il nome di una chat non esistente e il sistema ha comunque generato una nuova stanza ,mancanza del controllo sulle chat disponibili ]
\end{itemize}
\subsubsection{Visualizzazione dei dati}
\begin{itemize}
    \item Osservazioni: [Dettagli]
    \item Tempo impiegato: []
    \item Errori: [Dettagli]
    \item Commenti: [Dettagli]
\end{itemize}

\subsection{Soggetto 3}
\subsubsection{Leggere gli articoli}
\begin{itemize}
    \item Osservazioni: [Completato]
    \item Tempo impiegato: [$\simeq$ 18 secondi]
    \item Errori: []
    \item Commenti: [L'utente ha trovato gli articoli senza alcuna difficoltà]
\end{itemize}
\subsubsection{Uso del sistema interno di messaggistica}
\begin{itemize}
    \item Osservazioni: [Compltato]
    \item Tempo impiegato: [$\simeq$ 30 secondi]
    \item Errori: []
    \item Commenti: [L'utente ha trovato la chat senza alcun problema]
\end{itemize}
\subsubsection{Visualizzazione dei dati}
\begin{itemize}
    \item Osservazioni: [Fallito]
    \item Tempo impiegato: [$\simeq$ 80 secondi]
    \item Errori: [1]
    \item Commenti: [L'utente non ha trovato la checkbox per ottenere il ruolo di esperto e sbloccare la sezione con i grafici]
\end{itemize}

\section{Analisi e Discussione}
I risultati del Test di Krug indica un'area di miglioramento, ovvero semplificare l'accesso ai grafici. La raccomandazione da parte dei 3 soggetti in generale è stata:
\begin{itemize}
    \item Di semplificiare l'accesso ai grafici inserendo un pulsante nella navbar o nella home page, al posto di inserirlo solamente nell'About Us.
\end{itemize}

\section{Conclusione}
Il Test di Krug ha fornito preziose intuizioni sull'usabilità del sito web. L'implementazione delle modifiche raccomandate dovrebbe migliorare l'esperienza e la soddisfazione degli utenti.
\end{document}
